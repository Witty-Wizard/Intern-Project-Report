I got the opportunity to do my internship at Statcon Electronics India Ltd.
Statcon Electronics established in 1986 is one of India's largest ISO
9001-20151 certified manufacturer of Static energy Conversion system. During my
internship at Satcon Electronics India Ltd, Noida, I was first assigned to
explore Space Vector Modulation technique and then make a simulation for a
front end converter using Space Vector Modulation. I was part of the Embedded
Software team and my task was to develop a control algorith for the front end
converter.\\

The aim of the project to design a Active Front End converter for a Hybrid
inverter,aiming to enhance its efficiency and performance while reducing
harmonic distortions and improving the power factor.

\section{Timeline}
My internship at Statcon Electronics India Ltd started on 10th January, 2024.
My mentor gave me the task of exploring Space Vector Modulation and developing
control algorithm for the front end converter. This timeline will briefly
explain the course of my internship from beginning to end.

\subsection{January}
In January, I started learning about Space Vector Pulse Width Modulation
(SVPWM). This involved understanding two important things: the Clarke and Park
transforms. These transforms help to change three-phase voltages into a simpler
two-dimensional form, making it easier to control three-phase systems. The
Clarke transform turns three-phase voltages into two-phase parts, while the
Park transform changes these parts into a fixed frame of reference. I also
learned about space vectors, which show how the three-phase voltages combine.
This knowledge helped me understand SVPWM better. It's a technique that uses
space vectors to control how inverters produce voltage. This exploration gave
me a better grasp of how SVPWM works and its uses in power electronics.

\subsection{Feburary}
During February, I immersed myself in the study of the Clarke and Park
transforms, essential mathematical tools used in analyzing three-phase
electrical systems.The Clarke transform simplifies three-phase voltages into
two-phase components, making it easier to understand and analyze complex
electrical systems. Similarly, the Park transform further refines these
components into a fixed frame of reference, streamlining the analysis and
control of electrical signals. Alongside theoretical exploration, I practically
experimented by developing Python code to implement and test these
transformations. This hands-on approach not only deepened my understanding of
the transforms but also provided valuable insights into their real-world
applications and implementation challenges.

\subsection{March}

During March, I worked on developing and testing a three-phase Phase-Locked
Loop (PLL) algorithm using Python. This algorithm used the Clarke and Park
transforms to precisely estimate the angle of the space vector in the
three-phase system. Following the initial testing phase of the PLL algorithm in
Python, I transitioned to Simulink/MATLAB for further simulation. Within the
Simulink environment, I used MATLAB function blocks to implement the PLL
algorithm. Once this was completed, I proceeded to make a three-phase inverter
in Simulink using IGBT components. Leveraging MATLAB function blocks and the
output of the PLL, I then generated gate timing signals for the Front end of
the inverter, ensuring precise control and synchronization of the electrical
signals.

\subsection{April}

In April, I worked toward fine-tuning the Proportional-Integral (PI) controller
for the Phase-Locked Loop (PLL) using the Ziegler–Nichols method. This involved
adjusting the parameters of the PI controller to optimize the performance of
the PLL algorithm. By employing the Ziegler–Nichols method, I aimed to achieve
stability and responsiveness in tracking and synchronizing the phase of
electrical signals effectively. Additionally, I implemented a control loop in
MATLAB function blocks to measure the current flowing from the front end to the
grid. This control loop allowed for the adjustment of the output voltage to
either draw or supply the desired current, depending on the system
requirements.

\subsection{May}
In May, I completed the simulation and proceeded with the actual implementation
of the front end on hardware. I began by selecting a suitable microcontroller
with DSP capability, opting for the STM32F411 microcontroller. Then, I
initiated the coding process, developing functions for voltage transformations,
PID controller, and Phase Lock Loop (PLL). Additionally, I integrated unit
testing using Unity to test individual functions of the code. Moreover, I
created a native target to provide sample signals to the code, generating
output graphs for viewing. This approach allowed me to test my code without the
need for a microcontroller.

\subsection{June}
In June, I moved from the Noida office to the Hyderabad office, where I
continued the development of both the code and hardware. While there, I
familiarized myself with the standard structure of code that needed to be
followed and proceeded with the ongoing development process.