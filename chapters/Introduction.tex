An active front-end (AFE) converter, also known as a Regen Inverter, is a grid
interface converter that transfers power between an energy source and the
utility grid, and between the utility grid and a load. It's a controllable
rectifier that can exchange power between AC and DC power in both directions,
and can regenerate power to the mains to reduce power costs.
\section{Problem Statement}
My team at Statcon Electronics India Ltd has been focusing on on enhancing the
efficiency and performance of front-end converters. Presently, most active
front-end converters utilize Silicon Controlled Rectifiers (SCRs) for
rectification, employing a conventional 6-pulse converter configuration. My
task was to explore and design a active front-end using Insulated Gate Bipolar
Transistors (IGBTs) as alternatives to SCRs in the rectification process.

The primary objective of this project is to investigate the feasibility and
advantages of using IGBTs instead of SCRs for rectification in active front-end
converters. Specifically, we aim to implement a Space Vector Pulse Width
Modulation (SVPWM) technique to control the IGBTs effectively. This technique
offers precise control over the switching patterns of the IGBTs, allowing for
optimized power conversion and reduced harmonic distortion.

Furthermore, the project involves the development of a sophisticated control
algorithm to manage the operation of the IGBTs. This algorithm must facilitate
seamless transition between pulling current from the grid (rectification mode)
and supplying current to the grid (regeneration mode). The control system
should ensure stable and efficient operation under varying load conditions
while maintaining compliance with grid standards and regulations.

\section{Overview}
To understand the operation and control of the active front-end (AFE) converter
utilizing Insulated Gate Bipolar Transistors (IGBTs), it's essential to
understand concepts such as Space Vector Modulation (SVM), Clarke Transform,
and Park Transform.

\subsection{Space Vector Modulation}
This technique, also known as Space Vector Pulse Width Modulation (SVPWM), is a
method used to generate the switching signals for the IGBTs in the AFE
converter. It offers precise control over the IGBT switching patterns by
synthesizing the desired output voltage as a combination of multiple voltage
vectors. SVM has eight space vectors, other space vectors are synthesized by
alternating active and zero vectors over a switching period.

\subsection{Clarke Transform}
The Clarke Transform is a mathematical tool used to convert the three-phase AC
currents and voltages into two-phase components. By decomposing the complex
three-phase signals into simpler two-phase representations, the Clarke
Transform simplifies the control and analysis of the AFE converter. This
transformation is particularly useful in applications where two-phase control
is more practical or where the control algorithms are designed based on
two-phase references.

\subsection{Park Transform}
Building upon the Clarke Transform, the Park Transform further simplifies the
control of the AFE converter by transforming the two-phase quantities into a
stationary reference frame. This transformation aligns the reference frame with
the voltage vector, enabling linear control strategies such as
proportional-integral (PI) controllers.

\section{Challenges}
I faced many challenges in developing a Active front-end converter. Firstly,
distinguishing between Sinusoidal Pulse Width Modulation (SPWM) and Space
Vector Pulse Width Modulation (SVPWM) posed confusion due to their perceived
similarity in operational principles. Secondly, reconciling the non-zero
resultant vector in SVPWM with the traditional understanding of three-phase
systems, where the vector sum equals zero, required conceptual clarification to
ensure accurate system design and analysis. Another challenge was comprehending
how rectification and regeneration can occur through the same IGBT bridge, as
it involved bidirectional power flow management. Additionally, not knowing how
to write code in MATLAB was another challenge, but learning it opened up new
possibilities for simulating and analyzing our AFE converter designs.

\section{Summary}
This chapter addressed the problem statement concerning the exploration of
active front-end converters (AFEs) and the transition from Silicon Controlled
Rectifiers (SCRs) to Insulated Gate Bipolar Transistors (IGBTs) for
rectification.It also highlighted various AFE converter topologies, providing
brief distinctions between them. It discussed the challenges and difficulties
encountered throughout this project, including understanding modulation
techniques, reconciling operational principles, managing bidirectional power
flow, and learning MATLAB coding.